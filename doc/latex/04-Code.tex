%%!TEX root = ./UserManual.tex
\chapter{Code}
\label{chap:code}

%%%%%%%%%%%%%%%%%%%%%%%%%%%%%%%%%%%%%%%%%%%%%%%%%%%%%%%%%%%%%%%%
% GenGeoPop
%%%%%%%%%%%%%%%%%%%%%%%%%%%%%%%%%%%%%%%%%%%%%%%%%%%%%%%%%%%%%%%%

\section{GenGeoPop}
\label{section:gengeopop}

\subsection{Background}
\label{subsection:background}

To explain the algorithms used for generating the geography of the countries and their respective population, we have to introduce some background concepts:

\begin{description}
    \item[ContactPool]:
        A pool of persons that potentially come in contact with each other.
        If they actually do and if a transmission of an infection occurs depends on the algorithms in the simulator.
        We distinguish different types of ContactPools associated with work, school, community and households.
        Note that there is a difference between the size of a school (for example 500 students) and the size of the ContactPools inside the school (20 students).
    \item[ContactCenter]:
        A group of one or more ContactPools of the same type.
        This allows for a single College to contain multiple classes for example.
        It is also used to contain a single ContactPool, for example in the case of a Household.
    \item[K-12 student]: Part of the population from 3 until 18 years of age that are obligated (at least in Belgium) to attend a school. Students that skip or repeat years are not accounted for.
    \item[College student]:
        Persons older than 18 and younger than 26 years of age that attend a higher education. For simplicity we will group all forms of higher eduction into the same type of ContactCenter, a College. A fraction of college students will attend a college ``close to home'' and the others will attend a college ``far from home''.
        Most higher educations don't last for 8 years, but using this number we compensate for changes in the field of study, doctoral studies, advanced masters and repeating a failed year of study.
    \item[Employable]:
        We consider people of ages 18 to 65 to be employable. A fraction of people between 18 and 26 will attend a college, and the complementary fraction will be employable.
    \item[Active population]:
        The fraction of the employable part of the population that is actually working. A fraction of these workers will work ``close to home'' and the complementary fraction will commute to a workplace further away.
    \item[Household profile]:
        The composition of households in terms of the number of members and their age is an important factor in the simulation. In this case the profile is not defined by the age of its members or fractions, but through a set of reference households. This set contains a sample of households which are representative of the whole population in their composition.
    \item[GeoGrid locations]:
        Our data only allows for a very limited geographical resolution. We have the longitude and latitude of cities and municipalities (a distinction we will not make), which we shall use to create GeoGrid locations for the domicile of the households.
        All households in the same location are mapped to the coordinates of the location's center. These locations with corresponding coordinates will form a grid that will contain the complete simulation area.
\end{description}

\subsection{Generating the geography}
\label{subsection:gengeo}
We start by generating the geographical component, a GeoGrid.
This contains locations with an id, name, province, coordinates and nominative population figures.
These locations contain multiple ContactCenters, like Schools and Households, which in turn contain ContactPools.
This structure is internally generated by several ``Generators" and will afterwards be used by ``Populators" to fill the ContactPools.
The different types of ContactCenters are created by a different partial generator for each type and added separately to the GeoGrid.
We construct the following types of ContactCenters:

\begin{description}
    \item[Households]:
        The number of households is determined by the average size of a household in the reference profile and the total population size.
        These generated households are then assigned to a location by a draw from a discrete distribution based on the relative population size in each location.
    \item[K-12 Schools]:
        These schools contain on average 500 students, with an average of 20 students per class, which corresponds to 25 ContactPools.
        The total amount of schools in the region is determined by the size of the population and the fraction of those people who attend a K-12 school.
        The assignment of the schools to the locations is analogous to that of households.
    \item[Colleges]:
        These contain an average of 3000 students and 150 students per ContactPool.
        They are exclusively assigned to the 10 biggest locations in terms of population size.
        Within those 10 locations we again use a discrete distribution based on the relative population of the city compared to the total population in these top 10 cities.
    \item[Workplaces]:
        The assignment of workplaces is analogous to that of households, but now we account for commuting information to determine the actual number of workers in a location.
        We find this amount by the working people who live there plus the people that commute to this location minus the ones that live there but commute away.
        A workplace contains on average 20 people.
    \item[Communities]:
        We create both primary and secondary communities, each containing 2000 persons from all ages and occupations.
        The assignment is again analogous to that of households.
\end{description}

\subsection{Generating the population}
\label{subsection:genpop}
After creating the structures that will allow people to come in contact with each other and spread possible infections, we have create the population itself and determine the different ContactPools they will be in.
The persons are created based on the Household profiles in the HouseholdPopulator.
Analogous to the creation of the ContactCenters, we have a partial populator that will populate its own type of ContactPool:

\begin{description}
    \item[Households]:
        To create the actual persons, we draw a random household from the list of sample households and use that as a template for the number of members and their ages.
        We simply do this for each household in the GeoGrid, since we already determined the locations and amount of households while generating the geography.
    \item[K-12 Schools]:
        We start by finding all schools within a 10km radius of the household location. If this list is empty, we double this searching radius until it's no longer empty.
        We then choose a random ContactPool from the combined list of ContactPools in the found schools, even if this ContactPool now has more students than average (20).
    \item[Colleges]:
        The students who study ``close to home" are assigned to a college in a way analogous to the assignment to K-12 schools.
        For the ones that study further from home we first determine the list of locations where people from this locations commute to and choose one of these locations by use of a discrete distribution based on the relative commuting information.
        After we have chosen a location, we randomly choose a ContactPool at a college in this location and assign it to the commuting student.
    \item[Workplaces]:
        We first decide if the person actually has a job based on the unemployment parameter.
        We assign a workplace to an active worker that works ``close to home" in an analogous way as the assignment of K-12 schools to students.
        For the commuting workers we use a technique analogous to that of the commuting college students.
    \item[Communities]:
        The communities we can choose from at a location is determined in an analogous way to the K-12 schools.
        In primary communities we simply choose a random ContactPool from the list of Communities for each person in a household.
        In secondary communities however, we assign complete households to the ContactPools instead of each person in the household separately to a ContactPool.

\end{description}

%%%%%%%%%%%%%%%%%%%%%%%%%%%%%%%%%%%%%%%%%%%%%%%%%%%%%%%%%%%%%%%%
% Travel
%%%%%%%%%%%%%%%%%%%%%%%%%%%%%%%%%%%%%%%%%%%%%%%%%%%%%%%%%%%%%%%%

\section{Travel}
\label{section:travel}
The persons in the simulation also travel within different regions of the simulation for work or recreation.
This can also be within the same region. You perform a simulation with multiple regions by using a config file based on ``run\_regions\_default.xml" or ``run\_regions\_import.xml", where currently the regions Flanders, Wallonia, The Netherlands, France and Germany are defined. Please note that the population sizes and fractions provided are not representative of the actual values.
By using the option ``traveller\_amount" we can set the proportion of the population that is travelling at any given time.
Likewise, the option ``traveller\_max\_duration" indicates the maximal duration of travel. The actual duration will be chosen using a uniform distribution between 1 day and the maximal duration.
Finally, the option ``traveller\_fraction\_work\_travel" indicates the fraction of people that are travelling for work as opposed to recreationally.
We also need to set the probability of travelling to a specific destination country from a given origin country, which can be different for work and recreational travel.
We do this in the CSV files provided in the options ``traveller\_data\_work" and ``traveller\_data\_recreation" respectively, which can be found in the ``data" folder.

After every day in the simulation, a person decides if it will start travelling based on a TravellerProfile, which contains all information provided by the aforementioned options.
In this TravellerProfile it is also decided how long the stay will be, if it is a work or recreation related trip and where the person is travelling to, based on a discrete distribution of possible regions provided in the aforementioned CSV files. Within this chosen region, we select a random ContactPool based on the type of travel and add the traveller to this pool for the duration, removing him/her temporarily from the original ContactPools.
In case of work related travel, the temporary ContactPool will be of the type ``Work" and for recreational travel it will be of the type ``PrimaryCommunity".
All the information of this trip and the original ContactPools, which have to be restored after the trip ends, will then be stored in a TravellerIndex.
This TravellerIndex is also responsible for returning travellers to their original state.
At the end of each day, it will use a priority queue to efficiently determine if this is the end date for some travels.

To support this functionality, we create a Population that consists of one GeoGrid for each region in the simulation. This allows us to easily loop over the complete population, spanning over multiple regions, and for example retrieve the total amount of infected persons. Multi-region support is only provided when using GenGeoPop to generate the population, since we use different GeoGrids to represent the different regions.
